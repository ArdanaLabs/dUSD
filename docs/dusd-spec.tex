\documentclass{article} % Use the custom invoice class (invoice.cls)
\usepackage{hyperref}
\usepackage{todonotes}
\setuptodonotes{inline}
\setlength{\parindent}{0em}

\title{dUSD specification}
\begin{document}

\maketitle

\section{Introduction}

\textbf{The purpose of this document is to let the dUSD team decide whether we are ready to deploy.}
It is a living document that serves as a forcing function for discussions
around, and agreement on, decisions. \\

The document starts with the ``Interactions'' section, a user-centric
perspective on what service we would like to offer.
Next come descriptions of requirements we have on those interactions in the
``Acceptance criteria'' section.
Any important implementation details that will be relevant for building such a
system are detailed in the section after that.
Finally, the ``Tests'' section provide a non-exhaustive list of the way we plan
to falsify whether we should deploy. \\

Note: Every subsection that contain `TBD' (``to be determined'') in the title is
not fully agreed upon yet.
Every subsection containing `TBC' (``to be confirmed'', by Ryan) in the title is
mostly agreed upon, but still requires last confirmation.
The other sections are considered law.

\section{Interactions}

\subsection{Discover market prices}

For the MVP version, we implement a simplified, centralized market price
discovery mechanism.
This mechanism involves only a single on-chain component, called the Price
Module (PM).
This module contains $48$ hours of price data, aggregated off-chain and put
on-chain by admin.

\subsubsection{Price module: Write}

Admin can update the price information in the PM's datum by adding one more
price point.
When doing so, any price points older than $48$ hours are removed.

\subsubsection{Price module: Read}

Anyone can use the price module as a read-only input UTXO to their transactions
in order to discover the last $48$ hours of price information, one price point
per hour.

% The V2 version is left here in comments, for now.
% \subsection{Discover market prices (V2)}
% 
% A price oracle is a timeline of UTXOs\footnote{
%   In the blockchain industry, people often talk about concepts like price
%   oracles and DEX pools. However, within the EUTXO system built by Cardano, the
%   fundamental concept we are dealing is the UTXO. This means that we are dealing
%   with a `state' of a price oracle as the fundamental concept, not the price
%   oracle itself. Now, Plutus scripts like price oracles, between being created
%   and destroyed, basically only allow to be updated and read. This means that
%   one can draw a `timeline' through a series of UTXO, each one disappearing into
%   a transaction that produces the next. This set of UTXOs is what is referred to
%   as ``the price oracle''.
% }
% which stores the price information of a given asset (e.g. ADA) in terms of a
% reference currency (USD).
% Each price oracle will store $48$ hours worth of price data, and be updatable
% once every few minutes.
% A number of price oracles will be created for each asset of interest.  The
% person creating each price oracle will connect it to a price feed coming from
% some CEX/DEX.
% Next, an Oracle Security Module (OSM) will be implemented.
% This is another ``timeline of UTXOs'', which allows admin to select a number of
% price oracles as trusted price feeds and combines their prices in a secure way.
% The resulting price is exposed to the blockchain ecosystem.
% 
% Note: In this document, off-chain entities containing information about the
% price will be referred to as price feeds (not price oracles), e.g. the Coinbase
% website.
% 
% Note: We will build price oracles both for ADA and dUSD.
% 
% \subsubsection{Price oracle: Read}
% 
% The OSM can read all the pricing data stored in the price oracle.
% 
% \subsubsection{Write}
% 
% The owner/creator of the price oracle can write a new price value to the oracle.
% This endpoint can be called once per hour.
% 
% \subsection{Oracle security module}
% 
% \subsubsection{Read the `current' price}
% 
% Anyone can use the OSM as a read-only input UTXO to discover the price
% information of the last $48$ hours of some currency (e.g. ADA or dUSD relative
% to USD).
% 
% \subsubsection{Update}
% 
% The OSM is updated once per hour.
% Anyone can do this by connecting the OSM to all accepted price oracles.
% 
% \subsubsection{Update list of accepted price oracles}
% 
% Admin/governance can update the list of price oracles accepted by the OSM.
% 
\subsection{Vault}

\subsubsection{Collect vault information}

As a user, I can collect information on my vaults, i.e. the vaults I created.

\subsubsection{Initialize vault}

As a user, I can open a new vault.

\subsubsection{Deposit}

As a user, I can deposit ADA into a vault, so that I may take out a loan of dUSD
(see below).

\subsubsection{Withdrawal}

As a user, I can withdraw ADA from a vault, as long as my collateralization
ratio is above the liquidation ratio.
In the frontend, a ``minimum collateralization ratio'' will be built that
prevents users from transacting with their vault in such a way that their
collateralization ratio drops below the minimum collateralization ratio.
The reason for this is that many people use the MakerDAO system as
experimentation and lose thousands of dollars on day one.
Power users can easily circumvent this by running our PAB directly. \\

Note: The liquidation ratio can only be changed when the protocol is updated.

\subsubsection{Take out loan}

As a user, I can take out a loan in dUSD if I have enough collateral.
There is a minimum size for dUSD loans, called the ``debt floor''\footnote{
  This is an on-chain mechanism, which means it is enforced on everyone.
}.

\subsubsection{Pay back loan}

As a user, I can pay back (part of) a loan in dUSD I created in one of my
vaults.
In doing so, I am paying a stability fee to the buffer.

\subsubsection{Liquidate vault}

A vault is considered in an unhealthy state if its collateralization ratio is
below the liquidation ratio, both according to the current price and the price
from one hour ago.
The reason for this last part (using two prices) is that the vault owner should
get an hour after a new price gets published to fix his vault before allowing
liquidators to jump into action.
In addition, if a vault becomes sick through a very temporary ADA crash, but
recovers within an hour, there is no reason to liquidate. \\

Any user can detect that a vault is in an unhealthy state, and buy part of the
collateral in the vault at an admin-set discount rate compared to the market
prices of both ADA and dUSD.
This discount rate is called the liquidation discount, and will initially be
$3\%$.
Let us go through an example to explain how liquidation works.
When buying e.g. $100$ USD worth of ADA, the ADA is offered at a price of $97$
dUSD.
Anyone can buy a part of the ADA at this discounted exchange
rate\footnote{
  There are two reasons we allow liquidators to buy a part of the collateral,
  rather than forcing them to buy all of it.
  \begin{enumerate}
    \item Not forcing liquidators to buy all collateral makes the system less
      complex. Once a (possibly partial) liquidation has occured, the vault is
      still just a vault. If the collateralization ratio is still too low, a
      second person can come along and liquidate some more. If not, there's no
      reason a second person should be able to liquidate further.
    \item Forcing liquidators to buy all collateral would slow down liquidation
      arbitrage, since it requires one liquidity arbitrageur to have enough dUSD
      to buy all the collateral at once.
      Liquidation arbitrage is something we want to stimulate.
      It also unnecessarily assures punishing the vault owner very harshly.
  \end{enumerate}
}.

As time goes on, there are two options:
\begin{enumerate}
  \item By selling collateral at a discount, the collateralization ratio rises
    above the liquidation ratio, and the vault returns to a healthy state.
  \item All the collateral is sold but some dUSD debt remains in the vault.
    This is an exceptional situation to be avoided, since it endangers the idea
    behind the protocol.
    It that can only be caused by a significant ADA price drop where liquidators
    don't act on time.
    The solution is dissolving the vault, and thereby the dUSD debt.
\end{enumerate}

In addition to the liquidation discount given to liquidators, the vault owner
will be charged a liquidation fee for allowing his vault to become unhealthy.
The liquidation fee is initially set to $10\%$.
This means that only a part of the dUSD paid by the liquidators (in the initial
setup, $90\%$), is used to pay back debt, burning the rest.

\subsection{Buffer}

The buffer is a UTXO responsible for holding the protocol's profits and launching 
surplus and debt auctions. \\

Let's start by going through some general information on how the surplus and
debt auctions will work.
First of all, the purpose of surplus and debt auctions is to keep the peg in the
short-term.
More mid- to long-term, the stability fee is corrected to ensure the buffer's
reserves aren't endangered, and there is plenty of buying and selling power for
dUSD on the market. \\

Now onto the structure of the auctions.
The way to keep the peg, is by allowing anyone to trigger surplus/debt auctions
whenever certain conditions are reached.
Here, a surplus auction is one where the buffer mints dUSD and sells it for ADA,
and a debt auction is one where the buffer sells ADA from its reserves to buy
and burn dUSD.
As mentioned before, the stability fee is used to set the peg in the mid- to
long-term, and to ensure the buffer's reserves statistically increase rather
than decrease. \\

As ADA is accumulated in the buffer, admin (and later: governance) can decide to
share the value with the stakeholders (i.e. DANA token holders).
This mechanism is still under development. \\

Proposal: The conditions mentioned above under which one can trigger a `regular'
surplus/debt auction (trading dUSD for ADA), are that dUSD diverges from the peg
by more than $1\%$.

\subsubsection{Trigger surplus auction}

Anyone can trigger a dUSD $\leftrightarrow$ ADA surplus auction when certain
conditions are met.
Proposal: The conditions are that the dUSD price module claims the dUSD price
lies above $1.01$ USD.

\subsubsection{Trigger debt auction}

Anyone can trigger a dUSD $\leftrightarrow$ ADA debt auction when certain
conditions are met.
Proposal: The conditions are that the dUSD price module claims the dUSD price
lies below $0.99$ USD.

\subsubsection{Trigger ADA sale auction}

Admin can trigger an ADA $\leftrightarrow$ DANA sale auction of a given size
i.e. a given amount of ADA from the buffer's reserves is sold for DANA.

\subsubsection{Trigger DANA sale auction}

Admin can trigger a DANA $\leftrightarrow$ ADA sale auction of a given size,
i.e. a given amount of DANA from the buffer's reserves is sold for ADA.

\subsection{Protocol Parameters Module}

The Protocol Parameters Module is a UTXO responsible for setting the protocol-wide parameters.

\subsubsection{Set liquidation discount}

The admin user can set the liquidation discount, through running a `curl'
command on the Ardana Tenant.

\subsubsection{Set liquidation fee}

The admin user can set the liquidation fee, through running a `curl' command on
the Ardana Tenant.

\subsubsection{Set stability fee}

The admin user can set the stability fee (in percentage per year), through
running a `curl' command on the Ardana Tenant.

\subsubsection{Set debt floor}

The admin user can set the debt floor (in dUSD), through running a `curl'
command on the Ardana Tenant.

\subsection{Update mechanism for the protocol}

One important acceptance criterium is that the protocol can be updated without
requiring users to update their vaults or dUSD becoming a new currency.

Any updates to the protocol will be decided on by admin, and later governance.

This is implemented by deferring all protocol logic to minting policies.

These minting policies are defined in the Config UTXO which stores a lookup table that is mutable by an authorised party.

The authorisation for changing the Config is defined by a minting policy referenced by the config.

This allows the mechanism for updating the protocol to itself be changed.

Initially the mechanism will be to check that the Tx is signed by an admin key.

The admin can in the future change this mechanism to a more decentralised mechanism of control.

\section{Acceptance criteria}

\subsection{Price module}

Off-chain and on-chain criteria:
\begin{itemize}
  \item Only admin can create a price module
  \item Only admin can write to the oracle
  \item Writing a price value removes price values older than 48 hours from the
    datum
%   \item Writing the price value cannot be done more than once per hour
  \item Reading the price data after writing a new price point successfully,
    works
  \item Anyone can read the price data
\end{itemize}

Price feed bot:
\begin{itemize}
  \item Always collects information from at least three sources successfully
  \item Running the systemd service makes the UTXO get updated by at least three
    sources at least once every two hours
\end{itemize}

% The version 2 price discovery mechanism is stored in these comments:
% \subsection{Price oracle}
% 
% \begin{itemize}
%   \item Only the owner can write to the oracle
%   \item Anyone can read the price data
%   \item Writing a price values removes outdated price values (older than 48
%     hours) from the datum.
% \end{itemize}
% 
% \subsection{Oracle security module}
% 
% \begin{itemize}
%   \item Admin (and only admin) can update the list of accepted price oracles
%   \item The OSM's price information cannot be updated without linking to all
%     accepted price oracles
%   \item The price oracles are read-only inputs to the ``update OSM'' endpoint
%   \item Reading the current price after updating the OSM's price information and
%     waiting for the given time delay, leads to the predicted `current' price.
%     This is the median over all price oracles of the price averaged over the
%     values in the last hour. (This smearing prevents flash drops on some
%     exchanges from making the OSM's result crash and many vaults liquidate.)
% \end{itemize}
% 
\subsection{Vault}

Each user should have access to a website at "www.vault.ardana.com".
This website must be able to connect to their wallet to find out who they are
and allow them to set up vaults and interact with their existing vaults.

Acceptance criteria:
\begin{itemize}
  \item Anyone can create a vault, through a button on the website
  \item Anyone can create multiple vaults
  \item Vaults owned by the same person are independent, i.e. one of the
    person's vaults being sick, doesn't influence his other vaults
  \item There is a page that automatically connects to the user's wallet and
    displays information about his vaults: How many vaults, and for each vault
    the amount of collateral, debt, collateralization ratio, and how healthy the
    vault is.
  \item The vault overview is correct, clear and aesthetic
  \item Users can deposit, withdraw, take out a loan and pay back a loan
    (through forms on the website) only if the collateralization ratio doesn't
    drop below the liquidation ratio\footnote{
      In the UX (frontend only!), the collateralization ratio is banned from
      dropping under the minimum collateralization ratio.}
  \item No transaction can update the timestamp in the vault in a way that's
    inaccurate by more than $12$ hours
  \item Users cannot take out a loan or pay one back that leaves the dUSD debt
    above $0$ but below the debt floor\footnote{
    The puprose of the debt floor is to ensure good debt is issued against a
    vault to incentivize a liquidator to liquidate it should it become
    under-collateralized.}
% Comment for Version 2:
%   \item Users cannot take out a loan that leaves the dUSD debt above the debt
%     ceiling\footnote{
%     The debt ceiling has two purposes:
%     \begin{enumerate}
%       \item Deprecating collateral types by lowering the debt ceiling to below
%         the current amount of issued debt
%       \item Stopping OSM timing attacks, where vault owners exploit the OSM time
%         delay during a price crash, issuing a massive amount of debt thus
%         cashing out on worthless collateral
%     \end{enumerate}}
% 
% Debt ceilings are not useful against OSM timing attacks, since users can
% simply create multiple vaults.

  \item Users cannot withdraw or take out a loan that drops their
    collateralization ratio below the liquidation ratio
  \item When going through a sequence of transactions, stability fees are
    calculated correctly
  \item Only the owner of the vault can create transactions other than
    liquidation
  \item Anyone can liquidate a sick vault
  \item Liquidation is possibly if and only if the vault is in an unhealthy
    state
  \item One can buy as much collateral as is needed to make the vault healthy
    again, and not more
  \item During liquidation, collateral is bought at discount compared to the
    most current price from the price module, where the discount is the
    liquidation discount (set by admin)
  \item $10\%$ of the dUSD paid by the liquidator, is burnt rather than used to
    pay back the debt. This liquidation fee is meant to disincentivize users
    from allowing their vaults to become sick.
  \item The amount of dUSD paid back (when paying back a loan) is smaller than
    (or equal to) the loan
  \item Unless in the case of resource contention, any vault is liquidated
    within $1$ hour of becoming sick
    \todo{Improve this acceptance criterium.
    How do we test whether our liquidation bot works appropriately?
    What about ``Any vault that becomes sick, is noticed by our liquidation
    script within X minutes''?}
\end{itemize}

\subsection{Buffer}

\begin{itemize}
  \item Proposal: Anyone can trigger a surplus or debt auction when the dUSD
    price deviates from USD by more than $1\%$
  \item When auctions can be triggered, they are triggered within minutes.
    Auctions cannot be triggered more than once per hour, to give the auction
    sizing algorithm time to adjust.
  \item The starting amount of ADA in the buffer is enough to sustain the
    protocol through hardships with high probability
    \todo{What does this mean in practice? \\
    Proposal: Put in a decent amount at initialization, and set up an endpoint
    in the validator to add more later on. \\
    The stability fee will have a strong influence on this.}
  \item The auctions are sized such that the dUSD price is controlled, i.e. it
    is eventually consistent with USD.
    This means that given any real world change in supply and demand, the dUSD
    price eventually converges back to within $1\%$ of USD.
    \todo{Phrase this criterium properly.
    How can we test if the auction sizes are chosen appropriately?}
  \item Admin and only admin can trigger ADA and DANA sale auctions
\end{itemize}

\subsection{Protocol Parameters Module}
\begin{itemize}
  \item Admin and only admin can set the protocol-wide parameters (stability
    fee, liquidation discount, liquidation fee and debt
    floor)
  \item If the protocol-wide parameters are changed, this change is immediately
    and correctly seen in liquidations and when taking out loans
    % Comment for Version 2: Add debt ceiling to the protocol-wide parameters
\end{itemize}  

\subsection{Update mechanism for the protocol (TBD)}

\section{Implementation details}

This section discusses what we need in order to implement the interactions
mentioned above.

\subsection{Price module}

The price module is a UTXO which can only be created by admin.
It contains in its datum a map from timestamps (POSIX time) to the price value.
An off-chain bot will be written which aggregates price information from at
least five sources (CEXes and DEXes) and applies the appropriate calculations on
this information in order to conclude what the most accurate current price is.
This current price will then be submitted as a transaction to the price module
using one of the admin keys. \\

Off-chain bot:
\begin{itemize}
  \item Haskell script that collects information and submits a transaction
  \item Before collecting any price information, the script reads out the
    relevant UTXO's datum and checks if it has been updated in the last hour. If
    so, the script terminates successfully right away.
  \item The script is ran once every five minutes through a systemd service
\end{itemize}

Price module:
\begin{itemize}
  \item The on-chain code will be written in Plutarch
  \item The off-chain code will be built as a PAB
  \item The datum will contain a map from timestamps (in POSIX time) to price
    values
  \item The script address will be dependent on three admin public
    keys\footnote{
      These admin keys can only be updated when the protocol is updated.
      This decision will be enforced by using the keys to determine the script
      address, rather than putting them into the datum.
    }
\end{itemize}

Note: The off-chain bot will combine the price feeds through two operations:
\begin{enumerate}
  \item For each price feed, average the price over the values from the last
    hour
  \item Take the median of the resulting values
\end{enumerate}

% \subsection{Price oracle}
% 
% Each price oracle will carry an NFT used as a unique identifier.
% In addition, each price oracle contains the public key associated with the
% wallet which created it, so that only the creator of the oracle can update it.
% The price oracle can be updated once every hour.
% We will start with $3-5$ price oracles. \\
% 
% The OSM can be updated once every hour, and this transaction can be created by
% anyone.
% When it does, it pulls in the price information from all accepted price oracles
% and applies two calculations to that feed.
% First, it only retains the $70\%$ of price oracles that have most recently been
% updated, to assure obtaining the most recent price available.
% Secondly, it takes the median of the resulting price values.
% 
% \todo{How will price oracle updates and the OSM updates be rewarded? With a
% slight fee every time, coming from the profits stored in the buffer? Or should
% using a price oracle or OSM cost money, also to dUSD protocol users?}
% 
% If the OSM has not been updated for at least two hours, it stops spreading price
% information, thereby shutting down vault liquidations as well as creating new
% loans until the OSM is updated.
% This is a measure which will only kick in in the case of very serious resource
% contention, in which case once the system starts up again, we want to ensure
% that updating the OSM is the first transaction that happens.
% 
% \todo{Should the price oracles and the OSM all be combined into one UTXO, or
% be split into separate UTXOs?}
% 
\subsection{Vault}

\begin{itemize}
  \item On-chain code: Plutarch
  \item Off-chain code: PAB, tested through Maeserat and ContractModel tests.
    This means that users will use the frontend to contact our remote wallet
    PAB\footnote{
      The following statement is still being verified by Dan Firth's research.
      The protocol implementation as outlined in this document requires us to
      run two separate PABs.
      One PAB, referred to as the ``admin PAB'', will be connected to a wallet
      running locally on the Ardana Tenant.
      This admin PAB will be used to sign and submit admin transactions.
      On the other hand, the ``remote PAB'' will use Cardano's remote wallet
      implementation in order to generate the unsigned tranasctions our users
      need and yield them.
      These unsigned transactions will be sent to the user, where locally
      running javascript will connect to their wallet in order to sign and
      submit the transactions.
    }, which will generate the unsigned transaction they need.
  \item Collecting information on one person's vaults as well as vaults in
    general, are implemented as off-chain endpoints within the PAB, to avoid
    having to build a separate backend for the dUSD system
  \item A vault is a UTXO which holds an NFT, as a unique ID, and the collateral
    assets
  \item The vault's address is dependent on the owner's public key, ensuring the
    key cannot be changed by any transactions
  \item The vault's address is dependent on the three admin keys, ensuring the
    vault is connected to the correct buffer and price modules
  \item The vault's datum consists of two pieces of information: A timestamp of
    its last update and the amount of dUSD debt (including stability fee)
    calculated at that time
  \item Liquidation is only allowed if the current price and the price of an
    hour ago both state that the current collateralization ratio is below the
    liquidation ratio.
    This is a measure taken to allow vault owners a chance to make their vaults
    healthy whenever the price of the collateral dropped.
  \item All transactions must have a ``txInfoValidRange'' that's less than $12$
    hours long, to provide sufficiently accurate temporal information. This is
    enforced on-chain. The timestamp extracted from any transaction is the the
    beginning of this interval.
  \item When a dUSD loan gets paid back, the dUSD coins get burned in the vault.
    This means that the stability fee is burned as well, which drives up the
    dUSD price and allows for surplus auctions (rewarding governance owners).
    Note that the burning of the stability fees isn't directly linked to an
    equal amount of dUSD being minted in the buffer.
    Instead they are linked indirectly, through the former driving up the dUSD
    price and hence triggering surplus auctions.
\end{itemize}

Notes:
\begin{itemize}
  \item We do not want to make it possible to combine multiple transactions into
    one, as MakerDAO does.
    This is too complex a feature for an MVP, and once Orbis launches, gas fees
    will be very low.
  \item The names of the NFTs carried in the buffer and price modules, will be
    used as arguments to the vault's Plutus script.
    This ensures the vaults connect with the correct protocol setup.
  \item The frontend code will set the minimum collateralization ratio, which
    will be adjustable by Ardana itself.
\end{itemize}

In addition to the vault itself, we need a liquidation bot in order to liquidate
sick vaults as quickly as possible.
This liquidation bot will be a Haskell script running once every $10$ minutes on
the Ardana Tenant, as a systemd service.
The script will connect to a separate contract instance (set up specifically for
this script) of the dUSD admin PAB to collect information on all dUSD vaults,
and determine which ones are sick\footnote{
  This will be done by querying the ChainIndex.
  \todo{Figure out how to do this efficiently.
    Can we specifically find the vaults by checking the ADA price module's
    history?
  }
}.
For each sick vault, the script will set up a transaction through the admin DEX.
This transaction connects the vault to a trusted DEX (initially: a Minswap DEX
trading dUSD for ADA), liquidating the vault to the extend necessary.
In brokering this transaction, the liquidation bot earns profits, which are
stored in the admin wallet and used for protocol development.

\subsection{Buffer}
The buffer's script address will be dependent on the same three admin public
keys as the price module. \\

Note: However we implement the auctions, we might want to prevent them from
updating the buffer's UTXO, to avoid resource contention? \\

`Regular` surplus/debt auctions (dUSD $\leftrightarrow$ ADA) are implemented
using an orderbook-style auction based on the price modules.
They can be triggered by anyone whenever the necessary conditions are met.
When they are triggered, the auction sizing algorithm determines how much
dUSD/ADA is sold.
That amount is put into a separate UTXO, which allows anyone to buy the tokens
at a price determined by the ADA price module, where we act as if dUSD = USD in
value.
In doing so, whenever dUSD is trading above peg, we are selling dUSD at a
discount, and whenever dUSD is trading below peg, we are selling ADA at a
discount.
This methodology also prevents us from overcorrecting by launching auctions
which are too big. \\

The ADA/DANA sales follow a similar model, using the ADA and DANA price modules,
while giving a pre-set discount on the sale.
\todo{Pick this discount. $5\%$?}
\todo{Should this discount be set by the protocol, as an admin parameter, or set
separately for each ADA/DANA sale?}

If any surplus/debt auction or ADA/DANA sale is not fulfilled within a given
timeframe, the auction expires and an endpoint opens up to merge the tokens back
into the buffer.
\todo{Set the timeframe.}
\todo{Should this timeframe be set by the protocol, as an admin parameter, or set
separately for each auction/sale?}

The main advantage of this price modules-based approach, is that we always sell
at a trusted price, without allowing as many opportunities for manipulations.
(Classic auctions are vulnerable for manipulations through resource contention,
DEXes through executing big transactions, e.g. sandwich attacks.)
 
\subsection{Protocol Parameters Module}

The protocol parameters module will contain in its datum a map from timestamps to the stability fee
set at that time.
We use a map to make the stability fee calculations as accurate as
possible.\footnote{
  Imagine you open a vault, take out a loan and then leave the vault alone for
  two years.
  Just before you pay back the loan, governance decides to increase the
  stability fee.
  We want to use the stability fee that was applicable to the two year period
  when calculating your stability fee wherever possible, rather than
  retro-actively increasing your debt.
}. \\

We put together this map by figuring out the time of execution of the
transaction that changes the stability fee, within that transaction itself.
This is done by setting the timestamp to be the beginning of the
``txInfoValidRange'' range\footnote{
  \url{https://playground.plutus.iohkdev.io/doc/haddock/plutus-ledger-api/html/Plutus-V1-Ledger-Contexts.html}},
where the on-chain validator enforces this range to be less than $12$ hours
long. \\

In order to prevent datum overflow\footnote{
  This means we want to prevent the datum from becoming too big}
we limit the size of the map to $20$ data points.
This means that any transaction that changes the stability fee, will both add a
new entry to the map, and remove the oldest entry in case there are more than
$20$ entries. \\

The protocol parameters module will be a UTXO that includes in its datum the following:
\begin{itemize}
  \item Map from timestamps to stability fees
  \item Liquidation discount
  \item Liquidation fee
  \item Liquidation ratio
  \item Debt floor
\end{itemize}

\section{Tests}

\subsection{Contracts}

\subsubsection{Data Definitions}
Pseudo-code for Protocol Data Layouts. TODO keep up to date.

type Configuration = [CurrencySymbol]

type PricePoint = (Time,ADA/USD,DUSD/ADA)

type PriceModule = Vec 48 PricePoint

data Vault =
  Vault {
    debt :: Integer
  , authorisation :: CurrencySymbol
  }

type Size = Integer
type Price = Integer

data Buffer =
  Buffer {
    surplusTrigger :: Maybe (Size,Price)
  , debtTrigger :: Maybe (Size,Price)
  , liquidationDiscount :: Integer
  , liquidationFee :: Integer
  , stabilityFee :: Integer
  , debtFloor :: Integer
  }

\subsubsection{NFT minting policy}
This policy will be responsible for minting NFTs that identify the Configuration, Price Oracle, and Buffer UTXOs.

It is parameterised by a TXOutRef and only allows minting of its token when that TxOut is spent.

The tests must check that the policy
  - only validates if the TxOut it expects is being spent.
  - only mints a single token.
  - the output is otherwise unconstrained.

\subsubsection{Configuration Management Minting Policy}
Initially the Configuration will be managed by the owner of a private key.

This minting policy will check that the owner of the private key has signed the transaction and if so allow modification of the Configuration UTXO.

type Configuration = [CurrencySymbol]

The Configuration Datum is a list of CurrencySymbols.

The Main Script Redeemer is an index into this list and ensures that exactly one of the referenced token (CurrencySymbol,"") is minted.

The Configuration Management Minting Policy will always be at index 0.

This allows for Configuration Management Minting Policy itself to be swapped for a different policy.

The tests must check that the policy
  - only validates if the Tx is signed by the correct key.
  - only validates if the Configuration UTXO is present in the input and is returned to the Main Script address in the output.
  - mints exactly one token if these conditions are met.

\subsubsection{Main Script Validator}
There is only one script address for the dUSD protocol. It will defer all application logic to minting policies defined by the Configuration.

The Redeemer is an Integer index into the Configuration which is a list of CurrencySymbols.

The Redeemer looks up the Configuration UTXO in its inputs by looking for the input that carries the Configuration NFT.

It then ensures that the referenced token (CurrencySymbol,"") is minted.

The tests must check that the validator
  - only validates if the Configuration UTXO is present at some input
  - only validates if the Referenced Minting Policy is invoked

\subsubsection{Price Module Management Minting Policy}
The price module UTXO will carry an NFT which identifies the UTXO as valid.

This will be minted with the same NFT minting policy as for the Config but using a different unique TxOut.

Updating the Price Module will be done via a Minting Policy referenced by the Config. This allows its behaviour to be changed.

Initially this Minting Policy will simply check that the Tx is signed by an authorised key.

The tests must check that the policy
  - only validates if the Tx is signed by the correct key.
  - only validates if the Price Module UTXO is present in the input and is returned to the Main Script address in the output.
  - only validates if the price vector is treated as a bounded queue
    - a new price can be pushed onto the leading end
    - if the length is 48 then the tailing end must drop a price point
  - mints exactly one token if these conditions are met.

\subsubsection{Simple Vault Authorization Minting Policy}
This policy is parameterised by a PubKey. The user will create the policy as an access control to their vault.

More complex policies may be designed such as multi-party signature checks but this will likely be the common case.

This minting policy can be used in a Vault Datum to manage vault access.

The tests must check that the minting policy
  - only validates if the Tx is signed by the Key


\subsubsection{Vault Close Minting Policy}
Vaults can be opened without invoking any protocol logic. The user simply creates an empty vault datum and sends this to the main script address.

Closing a vault is done by invoking the Vault Close Minting Policy indexed via the Configuration.

This allows a Vault UTXO to be spent without returning to the Protocol address.

The tests must check that the minting policy
  - only validates if the Tx is authorised to manipulate the vault
    - the Vault contains a reference to a user defined minting policy that controls access to the vault
    - this minting policy must mint exactly one of its authorisation token
  - only validates if the vault is in a position to be closed
    - i.e. it has no outstanding debt
  - only validates if the vault does not return to the protocol address
    - i.e. the user can retrieve the funds in the vault and send them to any address
  - mints exactly one of the token

\subsubsection{Vault Change Ownership Minting Policy}
Vault ownership is determined by a Minting Policy referenced by the Vault UTXO.

This allows ownership to be programmable by the Vault owner.

The tests myst check that the minting policy
  - only validates if the Tx is authorised to manipulate the vault
    - the Vault contains a reference to a user defined minting policy that controls access to the vault
    - this minting policy must mint exactly one of its authorisation token
  - only validates if the only thing that changes about the vault is the ownership policy.
  - only validates if the vault returns to the protocol address
  - mints exactly one of the token


\subsubsection{Vault Deposit ADA Minting Policy}
Depends on the Configuration as a read only input.

Depositing ADA into a Vault can either be done on Vault creation which requires no direct protocol interaction or into an existing vault.

Depositing ADA into an existing Vault invokes the Vault Deposit Minting Policy.

The tests must check that the minting policy
  - only validates if the Tx is authorised to manipulate the vault
    - the Vault contains a reference to a user defined minting policy that controls access to the vault
    - this minting policy must mint exactly one of its authorisation token
  - only validates if the amount of ADA in the vault is increased
  - only validates if the vault returns to the protocol address with all other vault data unchanged
    - the transaction may not change the vault access control script or the Debt associated with the vault
  - mints exactly one of the token

\subsubsection{Vault Withdraw ADA Minting Policy}
Depends on the price module UTXO, Buffer, and the Configuration.
These should be read only inputs.


The tests must check that the minting policy
  - only validates if the Tx is authorised to manipulate the vault
    - the Vault contains a reference to a user defined minting policy that controls access to the vault
    - this minting policy must mint exactly one of its authorisation token
  - only validates if the amount of ADA in the vault is decreased
  - only validates if the new collateralization ratio is above the liquidation ratio
  - only validates if no other vault parameters are changed
    - the vault access script may not change
    - the dUSD debt may not change
  - mints exactly one of the token.

\subsubsection{Vault Take dUSD Loan Minting Policy}
Depends on the price module UTXO, Buffer, and the Configuration.
These should be read only inputs.

The tests must check that the minting policy
  - only validates if the Tx is authorised to manipulate the vault
    - the Vault contains a reference to a user defined minting policy that controls access to the vault
    - this minting policy must mint exactly one of its authorisation token
  - only validates if the amount of dUSD loaned out is increased
  - only validates if the new collateralization ratio is above the liquidation ratio
  - only validates if no other vault parameters are changed
    - the vault access script may not change
    - the ADA collateral may not change
  - mints exactly one of the token.

\subsubsection{Vault Pay back dUSD Loan Minting Policy}
Depends on the the Configuration.
This should be a read only input.

This does not need to check the Price Module.

The tests must check that the minting policy
  - only validates if the Tx is authorised to manipulate the vault
    - the Vault contains a reference to a user defined minting policy that controls access to the vault
    - this minting policy must mint exactly one of its authorisation token
  - only validates if the amount of dUSD loaned out is decreased
  - only validates if no other vault parameters are changed
    - the vault access script may not change
    - the ADA collateral may not change
  - mints exactly one of the token.

\subsubsection{Vault Liquidation}
Depends on the price module UTXO, Buffer, and the Configuration.
These should be read only inputs.

The tests must check that the minting policy
  - only validates if the Tx is authorised to manipulate the vault
    - the Vault contains a reference to a user defined minting policy that controls access to the vault
    - this minting policy must mint exactly one of its authorisation token
  - only validates if the liquidation is legitimate
    - this must follow the strict set of rules laid out in section 2.2.7
  - only validates if no other vault parameters are changed
    - the vault access script may not change 
  - mints exactly one of the token.

\subsubsection{Buffer Surplus Sale Minting Policy}
Depends on the Price Module UTXO and the Configuration.

The price module UTXO will carry an NFT which identifies the UTXO as valid.

This will be minted with the same NFT minting policy as for the Config but using a different unique TxOut.

Buffer actions are defined in minting policies referenced by the Config allowing their behaviour to be changed.

The tests must check that the policy
  - only validates if the dUSD price reported by the Price Module lies above 1.01 USD
  - only validates if the Buffer UTXO is present in the input and is returned to the Main Script address in the output.
  - only valdates if no other properties of the buffer are changed
    - this policy may not alter any buffer settings
  - only validates if the surplus sale trigger is set back to Nothing
  - mints exactly one token if these conditions are met.


\subsubsection{Buffer Debt Sale Minting Policy}
Depends on the Price Module UTXO and the Configuration.

The tests must check that the policy
  - only validates if the dUSD price reported by the Price Module lies below 0.99 USD
  - only validates if the Buffer UTXO is present in the input and is returned to the Main Script address in the output.
  - only valdates if no other properties of the buffer are changed
    - this policy may not alter any buffer settings
  - only validates if the debt sale trigger is set back to Nothing
  - mints exactly one token if these conditions are met.


\subsubsection{Simple Buffer Configuration Authorization Minting Policy}
Depends on the Configuration.

This allows manipulation of the buffer settings by an authorized party.

Initially this will done via admin key authorisation.

Buffer settings include
  - ADA sale trigger - this authorises an ADA sale at a specific size and price
  - DANA sale trigger - this authorises a DANA sale at a specific size and price
  - liquidation discount
  - liquidation fee
  - stability fee
  - debt floor

The tests must check that the policy
  - only validates if the admin key has signed the transaction
  - only validates if the Buffer UTXO is present in the input and is returned to the Main Script address in the output.
  - only validates if the collateral stored in the buffer is unchanged
  - only validates if at most one of the surplus/debt sale triggers is not Nothing
  - mints exactly one token if these conditions are met.




% \section{Remarks}
% 
% - Test for Cardano network congestion
% - Test what happens if someone hacks one or more price oracles
% - Other attack vectors which are not specific to our application, should be
%   listed and tested for

\section{To be determined}

\begin{itemize}
  \item Liquidation ratio upon initialization
  \item Liquidation fee upon initialization
  \item Liquidation discount upon initialization
  \item Minimum collateralization ratio upon initialization
  \item Debt floor
\end{itemize}

\end{document}
