\documentclass{article}
\usepackage{hyperref}
\usepackage{todonotes}
\setuptodonotes{inline}
\setlength{\parindent}{0em}

\title{Test plan: Hello World Deployment}
\begin{document}

\maketitle

\section{Interactions}

\begin{itemize}
  \item UTXO with a datum that is an int
  \item Website shows the int
  \item Website lets you increment the int
\end{itemize}

\section{Acceptance Criteria}

\begin{itemize}
  \item Anyone can increment the int (no authentication)
\end{itemize}

\section{Implementation Details}

\begin{itemize}
  \item This all happens in the dUSD repo
  \item On-chain code written in Plutarch
  \item \todo{How does one connect Plutarch code with PAB/cardano-browser-tx?}
  \item Off-chain code:
  \begin{itemize}
    \item Problem with pab: transaction balancing can currently only happen with wallet other than user wallet. If we pick the PAB, set up a plan for balancing transactions.
    \item Problem with pab: unmaintained (?(!))
    \item Problem with cardano-browser-tx: it compiles to JS
    \item Problem with cardano-browser-tx: it’s not haskell, so some transpilation of types will need to happen
    \item \todo{PAB vs cardano-browser-tx meeting}
    \begin{itemize}
      \item Ben Hart
      \item Dan Firth
    \end{itemize}
  \end{itemize}
  \item Front-end
  \begin{itemize}
    \item Static and comes from cloudflare
    \item Mostly html/css, some typescript as necessary
    \item User wallet: let’s try Nami and see if that works, then consider supporting Daedalus and Yoroi.
  \end{itemize}
\item Once we’re done, we can `git tag` the commit where that’s done and then upgrade this to the price module implementation in-place.
\end{itemize}

\section{Tests}

\begin{itemize}
  \item Apropos-tx testing for the on-chain code
  \begin{itemize}
    \item \todo{Spec for behaviour}
  \end{itemize}
  \item Unit test for on-chain code:
  \begin{itemize}
    \item increment from 0 to 1
    \item increment from 1 to 2
  \end{itemize}
  \item Maesarat
  \begin{itemize}
    \item Given an end-to-end test, Maesarat will benchmark the test and return the results. We can assert that certain time/memory requirements are met.
  \end{itemize}
  \item \todo{end-to-end testing including the front-end in nixos test}
  \begin{itemize}
    \item \todo{selenium?}
    \item \todo{Lighthouse, to check website performance?}
  \end{itemize}
  \item Contract Model Tests with PAB
  \begin{itemize}
    \item Wallet signs and submits a transaction
    \item Transactions are submitted by separate wallets (i.e. determine atomicity constraints)
  \end{itemize}
  \item \todo{end-to-end tests with the Cardano test net?}
  \item \todo{actually deploy to main net and run manual tests?}
\end{itemize}

\end{document}
